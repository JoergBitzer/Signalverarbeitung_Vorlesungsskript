\section{Bonus: Fibonacci und Z Transformation}
Ein Beispiel zur Anwendung der z Trafo sind die sog. Fibonacci-Zahlen. Nehmen wir an, wir haben
ein Kaninchenpärchen, dass pro Monat ein weiteres Pärchen zeugt
und dieses neue Pärchen ist nach einem weiteren Monat
geschlechtsreif\footnote{Gendefekte durch Inzucht schließen wir an
dieser Stelle auch mal aus.} und zeugt ebenfalls wieder neue
Pärchen. Die Frage lautet nun, wie nimmt die Kaninchenpopulation
zu, wenn kein Kaninchen stirbt?

Schauen wir uns zunächst die Populationsentwicklung an:

\begin{center}
\begin{tabular}{l|c|c|c|c|c|c|c|}
Monat  & 0 & 1 & 2 & 3 & 4 & 5 & 6  \\\hline

Neu    & 0 & 0 & 1 & 1 & 2 & 3 & 5 \\\hline

Gesamt & 1 & 1 & 2 & 3 & 5 & 8 & 13 \\\hline
\end{tabular}
\end{center}

Diese Tabelle lässt sich auch durch folgendes rekursives System
ausdrücken und beschreibt die Fibonacci-Folge.
\begin{equation}
    y(k) = y(k-1)+y(k-2) + \delta(k)
\end{equation}

Für dieses Problem lässt sich nicht so leicht eine direkte
Formulierung finden. 


Die selbe Lösungsstrategie verfolgen wir jetzt für das Problem der
Fibonacci-Zahlen. Ausgehend von
\begin{equation}\label{eq:example2:FiboStart}
y(k) = y(k-1) + y(k-2) + \delta(k),
\end{equation}
multiplizieren wie beide Seiten mit der Summe $\sum_{k =
	-\infty}^{\infty}z^{-k}$ und führen die Abkürzung aus Gleichung
\ref{eq:Def:zTrafo} ein.
\begin{eqnarray}\label{eq:example2}
Y(z) &=& z^{-1}Y(z) + z^{-2}Y(z) + 1 \\
Y(z)\left(1- z^{-1} -z^{-2} \right) &=& 1
\end{eqnarray}
Umgestellt ergibt sich
\begin{equation}\label{eq:example2:zLoesung1}
Y(z) = \frac{1}{1-z^{-1} - z^{-2}} = \frac{z^{2}}{z^{2} - z^{1} -1}
\end{equation}

Um Gl. \ref{eq:Def:geometrischeReihe} anwenden zu können, muss im
nächsten Schritt die Gleichung mit Hilfe der
Partialbruchzerlegung in die folgende Form gebracht werden
\begin{equation}\label{eq:example2:zLoesung2}
Y(z) = \frac{A}{z-a} + \frac{B}{z-b}
\end{equation}

Dies ist leider so direkt nicht möglich (siehe
\ref{sec:An:Partialbruch}), deshalb muss zunächst die Ordnung des
Zählers verringert werden. Wir führen eine Hilfsgröße
\begin{equation}\label{eq:example2:yhilf}
\widetilde{Y}(z) = \frac{Y(z)}{z}
\end{equation}
ein und lösen dann
\begin{equation}\label{eq:example2:PartialZer}
\widetilde{Y}(z) = \frac{A}{z-a} + \frac{B}{z-b}.
\end{equation}

Nachdem wir
die Lösung von $\widetilde{Y}(z)$ kennen, ist die Lösung für $Y(z)
= z \widetilde{Y}(z)$.

Um die Unbekannten $a,b,A,B$ in Gl. \ref{eq:example2:PartialZer}
zu bestimmen, ist zunächst die Produktdarstellung für Polynome zu
bestimmen (Fundamentalsatz der linearen Algebra):
\begin{equation}\label{eq:example2:Produktloesung}
\widetilde{Y}(z) = \frac{z}{z^{2} - z^{1} -1}
=\frac{z}{(z-a)(z-b)}.
\end{equation}
Die Bestimmung der Unbekannten $a,b$ erfolgt über die
Nullstellenberechnung des Polynoms. Dies ist für quadratische und
kubische Polynome durch Formeln möglich. Für Polynome höherer
Ordnung ist ein numerisches Verfahren zur Lösung
notwendig\footnote{In Matlab können die Wurzeln eines Polynoms mit
	{\tt roots} bestimmt werden.}

Für unser Beispiel ergeben sich die beiden Nullstellen
\begin{equation}\label{eq:example2:LoesungNullstellen}
a = \frac{1}{2}+\sqrt{\frac{1}{4}+1} \quad \mbox{und} \quad b = \frac{1}{2}-\sqrt{\frac{1}{4}+1}
\end{equation}

Um jetzt $A$ zu berechnen müssen wir mit Gl.
\ref{eq:example2:Produktloesung} folgende Gleichung lösen
\begin{eqnarray}\label{eq:example2:partialA}
A &= &(z-a)\widetilde{Y}(z)\big|_{z = a}\\
& = &\frac{(z-a)z}{(z-a)(z-b)}\big|_{z = a}\\
& = &\frac{z}{(z-b)}\big|_{z = a}\\
& = &\frac{a}{(a-b)}\\
& =
&\frac{\frac{1}{2}+\sqrt{\frac{5}{4}}}{\left(\frac{1}{2}+\sqrt{\frac{5}{4}}-\left(\frac{1}{2}-\sqrt{\frac{5}{4}}\right)\right)}\\
& =
&\frac{\frac{1}{2}+\sqrt{\frac{5}{4}}}{2\sqrt{\frac{5}{4}}}\\
& = &\frac{\frac{1}{2}+\sqrt{\frac{5}{4}}}{\sqrt{5}}
\end{eqnarray}
und für
\begin{eqnarray}\label{eq:example2:partial1B}
B &= &(z-b)\widetilde{Y}(z)\big|_{z = b}\\
& = &\frac{(z-b)z}{(z-a)(z-b)}\big|_{z = b}\\
& = &\frac{z}{(z-a)}\big|_{z = b}\\
& = &\frac{b}{(b-a)}\\
& =
&\frac{\frac{1}{2}-\sqrt{\frac{5}{4}}}{\left(\frac{1}{2}-\sqrt{\frac{5}{4}}-\left(\frac{1}{2}+\sqrt{\frac{5}{4}}\right)\right)}\\
& =
&\frac{\frac{1}{2}-\sqrt{\frac{5}{4}}}{-2\sqrt{\frac{5}{4}}}\\
&=&\frac{\frac{1}{2}-\sqrt{\frac{5}{4}}}{-\sqrt{5}}
\end{eqnarray}
Nachdem wir jetzt alle Konstanten bestimmt haben, können wir nach
der Multiplikation mit $z$
\begin{eqnarray}\label{eq:example2:kLoseung1}
Y(z) &=& \frac{Az}{z-a} + \frac{Bz}{z-b}\\
& = &A\frac{1}{1-az^{-1}} + B\frac{1}{1-bz^{-1}}
\end{eqnarray}
mit der schon bekannten Lösung der geometrischen Reihe folgt also
\begin{equation}\label{eq:esample2:Loesung1k}
A\frac{1}{1-az^{-1}} \Rightarrow Aa^k\gamma(k)
\end{equation}
und
\begin{equation}\label{eq:esample2:Loesung2k}
B\frac{1}{1-bz^{-1}} \Rightarrow Bb^k\gamma(k)
\end{equation}
Und somit für die direkte Berechnung der Ausgangsfolge ohne
Rekursion
\begin{eqnarray}\label{eq:example2:kLoseungEnde}
y(k) &=& (Aa^k+Bb^k)\gamma(k)\\\nonumber
&=&\left(\left(\frac{\frac{1}{2}+\sqrt{\frac{5}{4}}}{\sqrt{5}}\right)\left( \frac{1}{2}+\sqrt{\frac{5}{4}}\right)^k    +
\left(\frac{\frac{1}{2}-\sqrt{\frac{5}{4}}}{-\sqrt{5}}\right)\left(
\frac{1}{2}-\sqrt{\frac{5}{4}}\right)^k\right)\gamma(k)
\end{eqnarray}

